\documentclass[11pt]{article}
% \usepackage[utf8]{inputenc}
%\usepackage[T1]{fontenc}
\usepackage{hyperref}
\usepackage{geometry}
\hypersetup{colorlinks=true, linkcolor=blue, filecolor=magenta, urlcolor=cyan,}
% \urlstyle{same}
\usepackage{amsmath}
\usepackage{amsthm}
\usepackage{amsfonts}
\usepackage{ulem}
\usepackage{amssymb}
\usepackage{bbm}
\usepackage{mathrsfs}
\usepackage{subcaption}
\usepackage{tikz}
% \usepackage[version=4]{mhchem}
%\usepackage{stmaryrd}
\usepackage{float}
\usepackage{booktabs}
%\usepackage{pgfplots}
\usepackage{annotate-equations}
% \pgfplotsset{compat=1.18}
\usepackage{xcolor}
\usepackage{multirow}
\usepackage{listings}
%\usepackage{bbold}
\usepackage{graphicx}
\usepackage[export]{adjustbox}
\usepackage{mdframed}
\usepackage{enumitem}
\usepackage{zhlipsum}
\newmdenv[
backgroundcolor=gray!20,
linecolor=black,
linewidth=2pt,
skipabove=10pt,
skipbelow=10pt
]{customquote}


\makeatletter
\makeatother
\hypersetup{hidelinks,
	colorlinks=true,
	allcolors=black,
	pdfstartview=Fit,
	breaklinks=true
}

\theoremstyle{definition}
\newtheorem{define}{\noindent \textbf{\textsc{Definition}}}
\newtheorem{theorem}{\noindent \sc Theorem}
\newtheorem{lemma}{\noindent \textbf{\textsc{Lemma}}}
\newtheorem{prop}{\noindent \textbf{\textsc{Proposition}}}
\newtheorem{cor}{\noindent \textbf{\textsc{Corollary}}}
\newtheorem{example}{\noindent \sc Example}
\newtheorem{remark}{\indent Remark}
\newtheorem{assume}{\noindent \textbf{\textsc{Assumption}}}

\title{PROGRESS ON \\
	Endogenous Production Networks under Supply Chain Uncertainty}
\author{Jinhua Wu}
\newgeometry{left=2cm,right=2cm}
\linespread{1.3}

\newcommand{\basepath}{F:/12004835/replication_package_final/replication_package_final/output_figures}

\begin{document}
	\maketitle
	\tableofcontents
	\newpage
	
	\section{Some Notes on Introdution Part}
	\subsection{Definitions}
	\paragraph{Production Network} A production network refers to a complex system of interconnected entities and processes involved in the production and distribution of goods and services. This network includes suppliers, manufacturers, distributors, retailers, and end customers, all working together to ensure the efficient flow of products from raw materials to finished goods. The production network encompasses various stages such as procurement, manufacturing, logistics, and sales, each playing a crucial role in maintaining the overall efficiency and effectiveness of the production process. Effective management of a production network can lead to improved productivity, cost savings, and competitive advantage.
	
	\paragraph{Domar Weights} Domar weights, named after the economist Evsey Domar, are used to measure the contribution of each sector to the overall economy. Specifically, in a production network, the Domar weight of a sector is the ratio of that sector's output to the total GDP. This weight reflects the relative importance of a sector in the economy, considering both direct and indirect contributions through the production network. The concept is crucial in understanding how shocks to different sectors can propagate through the economy and affect overall productivity and welfare.
	
	\paragraph{Rish-averse representative household} A risk-averse representative household is a theoretical construct used in economic models to represent the behavior of a typical household that prefers to avoid risk. This household supplies a fixed amount of labor and makes consumption decisions to maximize its utility, which depends on the consumption of various goods. The utility function used in the model typically exhibits constant relative risk aversion (CRRA), meaning the household's aversion to risk remains constant regardless of its wealth level. The household makes consumption decisions after uncertainty in the economy is resolved, facing a budget constraint based on the prices of goods and the household’s income. The risk aversion parameter $(\rho)$ in the utility function quantifies how much the household dislikes risk: a higher $(\rho)$ indicates greater risk aversion. The household's decisions influence the production network because firms take into account the household's preferences and risk aversion when making their own production and pricing decisions.
	
	\paragraph{TFP Process} The TFP process refers to the Total Factor Productivity process, which is a crucial component in understanding economic growth and production efficiency. TFP measures the efficiency with which labor and capital are used together in the production process. The TFP process involves both the endogenous and exogenous factors that affect productivity in different sectors of the economy.
	
	\paragraph{Risk exposure} Risk exposure refers to the extent to which an entity (such as a firm, household, or economy) is vulnerable to various types of risks that can affect its performance or stability. In an economic context, risk exposure often involves uncertainties related to price fluctuations, supply chain disruptions, productivity shocks, and other external factors that can impact costs, revenues, and overall economic welfare.
	
	Variance of Unit Costs: Firms prefer inputs with stable prices and avoid techniques relying on inputs with positively correlated prices. This helps in diversifying risk and minimizing cost volatility.
	
	Correlation with Productivity Shocks: Firms prefer inputs whose prices are positively correlated with their productivity shocks. This means that during a negative shock, input prices are likely to be low, reducing expected cost increases.
	
	Risk-Adjusted Prices: Firms' technique choices are influenced by risk-adjusted prices, which account for the expected price of inputs and their covariance with the stochastic discount factor. Goods that are cheaper when aggregate consumption is low are particularly attractive.
	
	Impact on Supply Chain: Higher supplier volatility increases the likelihood of link destruction in supply relationships. Firms tend to move away from riskier suppliers to ensure stability.
	
	\paragraph{Hulten's Theorem} Hulten's theorem, named after economist Charles R. Hulten, is a fundamental result in the field of growth accounting and productivity analysis. The theorem states that the aggregate output (GDP) of an economy is a weighted sum of the outputs of its individual sectors, with the weights being the sectoral shares in total output. In simple terms, it implies that the proportional change in aggregate output is equal to the weighted sum of the proportional changes in the output of individual sectors.
	
	Mathematically, if $\Delta Y$ represents the change in aggregate output and $\Delta y_i$ represents the change in the output of sector $i$, Hulten's theorem can be expressed as:
	\begin{align*}
		\Delta Y = \sum\limits_i w_i \Delta y_i
	\end{align*}
	where $w_i$ is the Domar weight of sector $i$, reflecting its importance in the overall economy. It simplifies the analysis of how shocks to individual sectors affect the whole economy. It assumes a fixed production network, meaning the input-output relationships between sectors do not change in response to the shocks.
	
	\paragraph{Alternative Economy} An alternative economy refers to an economic system or a set of practices that differ from the traditional market-driven economy. It encompasses a wide range of economic models and activities that prioritize social, environmental, and ethical considerations over profit maximization. These alternative economic systems often emphasize community-oriented, cooperative, and sustainable practices.
	
	In the context of the provided document, alternative economies are used as benchmarks to evaluate the impact of various factors such as uncertainty on the production network and macroeconomic aggregates. Specifically, the document compares the baseline economy to alternative economies where firms are either unconcerned about risk when making sourcing decisions or have perfect foresight of productivity shocks. These comparisons help isolate the impact of uncertainty on the production network and its subsequent effect on GDP and welfare.
	
	\paragraph{Multi-sector economy} A multi-sector economy refers to an economic model that includes multiple sectors or industries, each producing different goods or services. This approach allows for a more detailed and realistic analysis of the economy by capturing the interactions and dependencies between various sectors. In a multi-sector economy, each sector may have its own production function, input requirements, and productivity shocks, and the outputs of some sectors serve as inputs for others, creating a complex network of interconnections.
	
	\paragraph{Productivity shifter} the productivity shifter is a function that represents how effectively a sector combines its inputs to produce output. It reflects the total factor productivity (TFP) of the sector, which varies depending on the chosen production technique $\alpha_i$. This shifter function is crucial in determining the productivity level of a sector and is influenced by the allocation of input shares among different suppliers.	
	
	\paragraph{Aggregate Risk} refers to the overall level of risk that affects the entire economy or a significant portion of it. It encompasses the uncertainties and potential fluctuations in economic variables that can impact multiple sectors simultaneously. Unlike idiosyncratic risk, which affects only individual firms or sectors, aggregate risk involves macroeconomic factors that can influence the entire economic system.
	
	\paragraph{Pareto Efficient Allocations} A Pareto efficient allocation is a state of resource distribution where it is impossible to make any individual better off without making at least one individual worse off. In other words, an allocation is Pareto efficient if no further reallocation can improve someone's situation without harming another person's situation. This concept is named after the Italian economist Vilfredo Pareto.
	
	
	
	
	
	
	
	
	
	\subsection{Summary for innovations} 
	\paragraph{Modeling Supply Chain Uncertainty} The authors construct a model of endogenous network formation to investigate how firms' decisions to mitigate supply chain risks affect the production network and macroeconomic aggregates. This model builds on and extends the work of Acemoglu and Azar (2020).
	
	\paragraph{Focus on Uncertainty}Unlike previous models that assume firms know the realization of shocks when choosing production techniques, this model incorporates uncertainty and beliefs about future productivity shocks into the decision-making process. This change allows the model to capture the impact of uncertainty on the structure of the production network.
	
	\paragraph{Technique Choice and Production Network} The model allows firms to choose production techniques that specify which intermediate inputs to use and how to combine them. These techniques can vary in terms of productivity, and firms can adjust the importance of suppliers or drop them altogether. This flexibility captures adjustments in the production network along both intensive and extensive margins.
	
	\paragraph{Risk-Adjusted Prices}
	Firms in the model choose techniques by considering risk-adjusted prices, reflecting the risk attitude of the representative household. This approach shows how aggregate risk and firms' sourcing decisions interact to shape the production network.
	
	\paragraph{Empirical Relevance}	
	The authors provide a basic calibration of the model using U.S. data to evaluate the importance of these mechanisms. They also highlight the model's ability to predict that increased uncertainty leads firms to prefer more stable suppliers, which reduces macroeconomic volatility but also lowers aggregate output.
	
	\paragraph{Comparative Analysis with Alternative Economies}
	The paper compares the baseline economy with alternative economies where firms either do not consider risk in their sourcing decisions or have perfect foresight of productivity shocks. This comparison helps to isolate the impact of uncertainty on the production network and macroeconomic outcomes.
	
	
	\section{Model}
	\subsection*{Notations and Symbols}
	\begin{table}[H]
		\centering
		\begin{tabular}{c|l}
			\hline
			\textbf{Notations} & \textbf{Meanings} \\
			\hline
			$\rho$ & The utility function quantifies how much the household dislikes risk\\
			$i\in\{1,\cdots,n\}$ & $n$ sectors \\
			$\mathcal{A}_i$ & The representative firm in sector $i$has access to a set of production techniques\\ 
			$\alpha_i = (\alpha_{i1},\cdots,\alpha_{in}) \in \mathcal{A}_i$ & Inputs used in production and combined in production \\
			$A_i(\alpha_i)$ & a productivity shitfer \\
			$L_i$ & Labor \\
			$X_i = (X_{i1},\cdots, X_{in})$ & A vector of intermediate inputs \\
			$\varepsilon_i$ & Stochastic component of sector $i$'s total factor productivity \\
			$\varepsilon\sim\mathcal{N}(\mu,\Sigma)$ & Collect the previous shock $\varepsilon = (\varepsilon_1,\cdots,\varepsilon_n)$ \\
			$\zeta(\alpha_i)$ & A normalization to simplify future expressions \\
			$\mathcal{A} = \mathcal{A}_1\times\cdots\times\mathcal{A}_n$ & Cartesian product \\
			$C = (C_1,\cdots,C_n)$ & consumption vector \\
			$u(\cdot)$ & CRRA with a coefficient of relative risk aversion $\rho\geqslant 1$ \\
			$P_i$ & the price of good $i$ \\
			$\Lambda$ & Stochastic discount factor \\
			$\overline{P}$ & Price index \\
			$\beta$ & consumption shares \\
			$K_i(\alpha_i, P)$ & The unit cost of production \\
			$Q_i$ & the equilibrium demand for good $i$ \\
			$\mathcal{L}(\alpha) = (I - \alpha)^{-1}$ & The Leontief inverse \\
			$\omega_i$ & Domar weight of sector $i$ \\
			$\alpha_i^*$ & a technique to maximize expected discounted profits \\
			$\lambda(\alpha^*)$ & stochastic discount factor \\
			$k_i(\alpha_i,\alpha^*)$ & The log of unit cost \\
			$\mathcal{R}(\alpha^*)$ & The vector of equilibirum risk-adjusted price \\
			\hline
		\end{tabular}
	\end{table}
	
	\subsection{Firms and production functions}
	\paragraph{The corresponding production function} 
	\begin{align}
		F(\alpha_i, L_i, X_i) = e^{\varepsilon_i} A_i(\alpha_i) \zeta(\alpha_i) L_i^{1 - \sum\limits_{j=1}^n \alpha_{ij}} \prod\limits_{j=1}^n X_{ij}^{\alpha_{ij}} \label{eq-1}
	\end{align}
	where $L_i$ is labor and $X_i = (X_{i1},\cdots, X_{in})$ is a vector of intermediate inputs. The term $\varepsilon_i$ is the stochastic component of sector $i$'s total factor productivity. Finally, $\zeta(\alpha_i)$ is a normalization to simplify future expressions.
	
	\paragraph{Set of feasible production techniques} 
	\begin{align*}
		\mathcal{A}_i = \left\lbrace \alpha_i \in [0,1]^n :\ \sum\limits_{j=1}^n \alpha_{ij} \leqslant \bar{\alpha}_i \right\rbrace
	\end{align*}
	where $0< 1 - \bar{\alpha}_i < 1$ provides a lower bound on the share of labor in the production of good $i$. 
	
	\begin{assume}
		$A_i(\alpha_i)$ is smooth and strictly log-concave.
	\end{assume}
	
	For each sector $i$, there is a unique vector of ideal input shares $\alpha_i^{\circ}\in\mathcal{A}_i$ that maximize $A_i$ and that represents the most productive way to combine intermediate inputs to produce good $i$. \textbf{We normalize $A_i(\alpha_i^{\circ}) = 1$ for all $i$.}
	
	\paragraph{Example} One example of a function $A_i(\alpha_i)$ that satisfies Assumption 1 is the quadratic form 
	\begin{align}
		\log A_i(\alpha_i) = \frac{1}{2} (\alpha_i - \alpha_i^{\circ})^{T} \bar{H}_i (\alpha_i - \alpha_i^{\circ}) \label{eq-2}
	\end{align}
	where $\bar{H}_i$ is a negative-definite matrix that is also the Hessian of log $A_i$.
	
	\subsection{Household preferences}
	\paragraph{CRRA}
	A risk-averse representative household supplies one unit of labor in elastically and chooses aconsumption vector $C = (C_1,\cdots,C_n)$ to maximize
	\begin{align}
		u\left(\left(\frac{C_1}{\beta_1}\right)^{\beta_1}\cdots \left(\frac{C_n}{\beta_n}\right)^{\beta_n}  \right) \label{eq-3}
	\end{align}
	where $\beta_i>0$ for all $i$ and $\sum\limits_{i=1}^n \beta_i = 1$. We refer to $Y = \prod\limits_{i=1}^n (\beta_i^{-1} C_i)^{\beta_i}$ as aggregate consumption or, equivalently in this setting, GDP. The utility function $u(\cdot)$ is CRRA\footnote{CRRA stands for Constant Relative Risk Aversion. It is a type of utility function used in economics to describe the behavior of agents who have a consistent attitude towards risk, regardless of their wealth level. The CRRA utility function is commonly used in models of consumer behavior, finance, and macroeconomics because it has several desirable properties, including scalability and tractability.} with a coefficient of relative risk aversion $\rho\geqslant 1$. The household makes consumption decisions after uncertainty is resolved and so in each state of the world it faces the budget constraint
	\begin{align}
		\sum\limits_{i=1}^n P_iC_i \leqslant 1 \label{eq-4}
	\end{align}
	where $P_i$ is the price of good $i$, and the wage is used as the numeraire.
	
	\paragraph{Stochastic discount factor} 
	Firms are owned by the representative household and maximize expected profits discounted by the household’s stochastic discount factor
	\begin{align}
		\Lambda = u^{\prime}(Y) / \overline{P} \label{eq-5}
	\end{align}
	wbere $\overline{P} = \prod\limits_{i=1}^n P_i^{\beta_i}$ is the price index. 
	
	\paragraph{Log GDP}
	From the optimization problem of the household it is straightforward to show that
	\begin{align}
		y = -\beta^T p \label{eq-6}
	\end{align}
	where $y = \log Y$, $p = (\log P_1,\cdots,\log P_n)$ and $\beta = (\beta_1,\cdots,\beta_n)$. Log GDP is thus the negative of the sum of log prices weighted by the consumption shares $\beta$. Intuitively, as prices decrease relative to wages, the household can purchase more goods, and aggregate consumption increases.
	
	\subsection{Unit cost minimization}
	\paragraph{The second stage problem}
	Under a given technique $\alpha_i$, the cost minimization problem of a firm in sector $i$ is
	\begin{align}
		K_i(\alpha_i, P) = \min\limits_{L_i, X_i} \left( L_i + \sum\limits_{j=1}^n P_j X_{ij} \right),\quad \text{subject to} \ F(\alpha_i, L_i, X_i) \geqslant 1 \label{eq-7}
	\end{align}
	
	\textcolor{blue}{Thus we construct a Lagrangian Function as:}
	\begin{align*}
		\color{blue} \mathcal{L} = L_i + \sum_{j=1}^{n} P_j X_{ij} + \lambda \left(1 - e^{\varepsilon_i} A_i(\alpha_i) \zeta(\alpha_i) \left(\prod\limits_{j=1}^{n} X_{ij}^{\alpha_{ij}}\right) L_i^{\left(1 - \sum\limits_{j=1}^{n} \alpha_{ij}\right)} \right)
	\end{align*}
	
	\textcolor{blue}{First-Order Conditions: Taking the first-order conditions with respect to \(L_i\), \(X_{ij}\), and \(\lambda\), we get:}
	\textcolor{blue}{\begin{align*}
	 	0 &= 1 - \left(1 - \sum\limits_{j=1}^{n} \alpha_{ij}\right)  e^{\varepsilon_i} \lambda A_i(\alpha_i) \zeta(\alpha_i) \left(\prod\limits_{j=1}^{n} X_{ij}^{\alpha_{ij}}\right) L_i^{\left(- \sum\limits_{j=1}^{n} \alpha_{ij}\right)} \\
		0 &= P_j - \lambda e^{\varepsilon_i} A_i(\alpha_i) \zeta(\alpha_i) L_i^{\left(1 - \sum\limits_{j=1}^{n} \alpha_{ij}\right)} \left(\prod\limits_{j=1}^{n} X_{ij}^{\alpha_{ij}}\right) X_{ij}^{- 1} \alpha_{ij} 
	\end{align*}}
	\textcolor{blue}{Thus we could get the following things:}
	\textcolor{blue}{\begin{align*}
			L_i &= \left(1 - \sum\limits_{j=1}^n \alpha_{ij} \right) \lambda \\
			X_{ij} &= \frac{\lambda \alpha_{ij}}{P_j}
	\end{align*}}
	\textcolor{blue}{Thus we could substitute to the equation and get the following:}
	\begin{align}
		K_i(\alpha_i, P) = \frac{1}{e^{\varepsilon_i} A_i(\alpha_i)} \prod\limits_{j=1}^n P_j^{\alpha_{ij}} \label{eq-8}
	\end{align}
	
	\subsection{Technique choice}
	\paragraph{The first stage problem} 
	Given an expression for $K_i$, the first stage of the representative firm’s problem is to pick a technique $\alpha_i\in \mathcal{A}_i$ to maximize expected discounted profits, that is,
	\begin{align}
		\alpha_i^* \in \underset{\alpha_i \in \mathcal{A}_i}{\arg\max}\ \mathbb{E}\ [\Lambda Q_i (P_i - K_i(\alpha_i, P))] \label{eq-9}
	\end{align}
	
	where $Q_i$ is the equilibrium demand for good $i$, and where the profits in different states of the world are weighted by the household’s stochastic discount factor $\Lambda$. The representative firm takes $P$, $Q_i$ and $\Lambda$ as given, and so the only term in (\ref{eq-9}) over which it has any control is the unit cost $K_i(\alpha_i, P)$.
	
	\subsection{Equilibrium conditions}
	\paragraph{Competitive Pressure} 
	In equilibrium, competitive pressure pushes prices to be equal to unit costs, so that
	\begin{align}
		P_i = K_i(\alpha_i, P) \quad \text{for all} \ i\in\{1,2,\cdots, n\} \label{eq-10}
	\end{align}
	
	\begin{define}
		An equilibirum is a choice of technique $\alpha^* = (\alpha_1^*,\cdots, \alpha_n^*)$ and a stochastic tuple $(P^*, C^*, L^*, X^*, Q^*)$ such that
		\begin{enumerate}[leftmargin=1cm, label=\arabic*.]
			\item (Optimal technique choice) For each $i\in\{1,2,\cdots,n\}$, the technique choice $\alpha_i^*\in\mathcal{A}_i$ solves (\ref{eq-9}) given price $P^*$, demand $Q^*_i$ and the stochastic discount factor $\Lambda^*$ given by (\ref{eq-5}).
			\item (Optimal input choice) For each $i\in\{1,2,\cdots,n\}$, factor demands per unit of output $L_i^*/Q_i^*$ and $X_i^*/Q_i^*$ are a solution to (\ref{eq-7}) given price $P^*$ and the chosen technique $\alpha_i*$.
			\item (Consumer maximization) The consumption vector $C^*$ maximizes (\ref{eq-3}) subject to (\ref{eq-4}) given prices $P^*$.
			\item (Unit cost pricing) For each $i\in\{1,2,\cdots,n\}$, $P^*_i$ solves (\ref{eq-10}) where $K_i(\alpha_i^*, P^*)$ is given by (\ref{eq-8}).
			\item (Market clearning) For each $i\in\{1,2,\cdots,n\}$,
			\begin{align}
				C_i^* + \sum\limits_{j=1}^n X_{ji}^* = Q_i^* = F_i(\alpha_i^*, L_i^*, X_i^*), \quad \text{and} \ \sum\limits_{i=1}^n L_i^* = 1 \label{eq-11}
			\end{align}
		\end{enumerate}
	\end{define}
	
	
	\section{Equilibirum prices and GDP in a fixed-network economy}
	\paragraph{Domar weight} 
	We also define the Domar weight $\omega_i$ of sector $i$ as the ratio of its sales to nominal GDP, such that
	\begin{align*}
		\omega_i = \frac{P_iQ_i}{P^TC}
	\end{align*}
	Also $\omega^T = \beta^T \mathcal{L}(\alpha) > 0$ in the model.
	
	\begin{lemma}
		Under a given network $\alpha$, the vector of log prices is given by
		\begin{align}
			p(\alpha) = -\mathcal{L}(\alpha)(\varepsilon + a(\alpha)) \label{eq-12}
		\end{align}
		and log GDP is given by
		\begin{align}
			y(a) = \omega (a)^T (\varepsilon + a(\alpha)) \label{eq-13}
		\end{align}
		where $a(\alpha) = (\log A_i(\alpha_i),\cdots, \log A_n(\alpha_n))$
	\end{lemma}
	
	
	\paragraph{The first and second moments}
	\begin{align}
		\mathbb{E}[y(\alpha)] = \omega(a)^T (\mu + a(\alpha)) \quad \mathbb{V}[y(\alpha)] = \omega (a)^T \Sigma\omega(\alpha) \label{eq-14}
	\end{align}
	
	
	\begin{cor}
		For a fixed production network $\alpha$, the following holds:
		\begin{enumerate}[leftmargin=1cm, label=\arabic*.]
			\item The impact of a change in expected TFP $\mu_i$ on the moments of log GDP is given by
			\begin{align*}
				\frac{\partial\mathbb{E}[y]}{\partial \mu_i} = \omega_i \quad \frac{\partial \mathbb{V}}{\partial\mu_i} = 0
			\end{align*}
			\item The impact of a change in volatility $\Sigma_{ij}$ on the moments of log GDP is given by
			\begin{align*}
				\frac{\partial\mathbb{E}[y]}{\partial \Sigma_{ij}} = 0 \quad \frac{\partial \mathbb{V}}{\partial\Sigma_{ij}} = \omega_i\omega_j
			\end{align*}
		\end{enumerate}
	\end{cor}
	
	
	\section{Firm decisions}
	\paragraph{Log of those things}
	Log of the stochastic discount factor
	\begin{align*}
		\lambda(\alpha^*) = \log\Lambda(\alpha^*)
	\end{align*}
	
	The log of the unit cost
	\begin{align*}
		k_i(\alpha_i,\alpha^*) = \log K_i(\alpha_i, P^*(\alpha^*))
	\end{align*}
	where $\alpha^*$ denotes the equilibrium network.
	
	\paragraph{Probelm of the firm}
	Using this notation, we can reorganize the problem of the firm (\ref{eq-9}) as
	\begin{align}
		\alpha_i^* \in \underset{\alpha_i\in\mathcal{A}_i}{\arg\min} \mathbb{E}[k_i(\alpha_i, \alpha^*)] + \text{Cov}[\lambda(\alpha^*), k_i(\alpha_i,\alpha^*)] \label{eq-15}
	\end{align}
	\textcolor{blue}{Combining the equation with (\ref{eq-5}) we can write $\lambda = \log(\Lambda)$ as}
	\begin{align*}
		\color{blue}\lambda(\alpha^*) = -(1-\rho) \sum\limits_{i=1}^n \beta_ip_i(\alpha^*)
	\end{align*}
	\textcolor{blue}{Taking the log of (\ref{eq-8}) yields}
	\begin{align*}
		\color{blue} k_i(\alpha_i,\alpha^*) = -(\varepsilon_i + a(\alpha_i)) + \sum\limits_{j=1}^n \alpha_{ij} p_j (\alpha^*)
	\end{align*}
	\textcolor{blue}{Both $\lambda(\alpha^*)$ and $k_i(\alpha_i,\alpha^*)$ are normally distributed since they are linear combinations of $\varepsilon$ and the
	log price vector, which is normally distributed by Lemma 1.}
	
	\textcolor{blue}{Turning to the firm problem 9, we can write}
	\begin{align*}
		\color{blue} \alpha_i^* \in \underset{\alpha_i\in\mathcal{A}_i}{\arg\min} \mathbb{E}\left[\Lambda\frac{\beta^T\mathcal{L}(\alpha^*)\mathbbm{1}_i}{P_i} K_i(\alpha_i, P) \right],
	\end{align*}
	\textcolor{blue}{where we have used (A.7) from Supplemental Appendix A in Kopytov et al.(2024). We can drop $\beta^T\mathcal{L}(\alpha^*)\mathbbm{1}_i>0$ since it is a deterministic scalar that does not depend on $\alpha_i$. Rewriting thisequation in terms of log quantities yields}
	\begin{align*}
		\color{blue} \alpha_i^* \in \underset{\alpha_i\in\mathcal{A}_i}{\arg\min} \mathbb{E}[k_i(\alpha_i, \alpha^*)] + \text{Cov}[\lambda(\alpha^*), k_i(\alpha_i,\alpha^*)]
	\end{align*}
	
	The objective function in (\ref{eq-15}) captures how beliefs and uncertainty affect the production network. Its first term implies that the firm prefers to adopt techniques that provide, in expectation, a lower unit cost of production. Taking the expected value of the log of (\ref{eq-8}), we can write this term as
	\begin{align*}
		\mathbb{E} [k_i(\alpha_i,\alpha^*)] = -\mu_i - a_i(\alpha_i) + \sum\limits_{j=1}^n \alpha_{ij}\mathbb{E}[p_j]
	\end{align*}
	
	\textcolor{blue}{Thus we could substitute $k_i(\alpha_i,\alpha^*)$ to the (\ref{eq-15}):}
	\textcolor{blue}{\begin{align*}
		 \mathbb{E}[k_i(\alpha_i,\alpha^*)] &= \mathbb{E}[\log K_i(\alpha_i,\alpha^*)] = \mathbb{E}[-\varepsilon_i - \log A_i(\alpha_i) + \sum\limits_{j=1}^n \alpha_{ij} P_j] \\
		& = -\mu_i - \eqnmarkbox[red]{1}{a_i(\alpha_i)} +  \sum\limits_{j=1}^n \alpha_{ij} \mathbb{E}[P_j] 
	\end{align*}}
	\annotate[yshift = -0.5em]{below, right}{1}{By the definition $a(\alpha) = (\log A_i(\alpha_i), \cdots, \log A_n(\alpha_n))$}
	
	so that, unsurprisingly, the firm prefers techniques that have high productivity ai and that rely on inputs that are expected to be cheap.
	
	The second term in (\ref{eq-15}) captures the importance of aggregate risk for the firm’s decision. It implies that the firm prefers to have a low unit cost in states of the world in which the marginal utility of consumption is high. As a result, the coefficient of risk aversion $\rho$ of the household indirectly determines how risk-averse firms are. We can expand this term as
	\begin{align*}
		\text{Cov}[\lambda,k_i] = \text{Corr}[\lambda, k_i]\sqrt{\mathbb{V}[\lambda]}\sqrt{\mathbb{V}[k_i]}
	\end{align*}
	which implies that the firm tries to minimize the correlation of its unit cost with λ$\lambda$. Furthermore, since prices and GDP tend to move in opposite directions (see Lemma 1), $\text{Corr}[\lambda, k_i]$ is typically positive, and so firms seek to minimize the variance of their unit cost. This has several implications for their choice of suppliers. To see this, we can use (\ref{eq-8}) to write
	\textcolor{blue}{\begin{align*}
		\mathbb{V}[k_i(\alpha_i,\alpha)] &= \mathbb{V}[\log K_i(\alpha_i,\alpha)] = \mathbb{V}[-\varepsilon_i - \log A_i(\alpha_i) + \sum\limits_{j=1}^n \alpha_{ij} P_j] \\
		& = \Sigma_{ii} + \sum\limits_{j=1}^n \alpha_{ij} \mathbb{V}[p_j] + \sum\limits_{j\neq k} \alpha_{ij}\alpha_{ik} \text{Cov}[p_j,p_k] + 2\text{Cov}\left[-\varepsilon_i,\sum\limits_{j=1}^n \alpha_{ij}p_j \right] 
	\end{align*}}
	
	\textcolor{blue}{Thus we could conclude:}
	\begin{align}
		\mathbb{V}[k_i(\alpha_i,\alpha)] = \sum\limits_{j=1}^n \alpha_{ij} \mathbb{V}[p_j] + \sum\limits_{j\neq k} \alpha_{ij}\alpha_{ik} \text{Cov}[p_j,p_k] + 2\text{Cov}\left[-\varepsilon_i,\sum\limits_{j=1}^n \alpha_{ij}p_j \right]  +  \Sigma_{ii} \label{eq-16}
	\end{align}
	
	\begin{lemma}
		In equilibirum, the technique choice problem of the representative firm in sector $i$ is
		\begin{align}
			\alpha_i^* \in \underset{\alpha_i\in\mathcal{A}_i}{\arg\max} a_i(\alpha_i) - \sum\limits_{j=1}^n \alpha_{ij}\mathcal{R}_j(\alpha^*) \label{eq-17}
		\end{align}
		where
		\begin{align}
			\mathcal{R}(\alpha^*) = \mathbb{E}[p(\alpha^*)] + \text{Cov}[p(\alpha^*),\lambda(\alpha^*)] \label{eq-18}
		\end{align}
		is the vector of equilibirum risk-adjusted price, and where
		\begin{align*}
			\mathbb{E}[p(\alpha^*)] = -\mathcal{L}(\alpha^*) (\mu + a(\alpha^*)) \quad \text{Cov}[p(\alpha^*),\lambda(\alpha^*)] = (\rho-1)\mathcal{L}(\alpha^*)\Sigma [\mathcal{L}(\alpha^*)]^T\beta
		\end{align*}
	\end{lemma}
	
	\paragraph{First-order Condition} 
	Se can take the first-order condition for an interior solution of problem (\ref{eq-17}) and use the implicit function theorem to write
	\begin{align}
		\frac{\partial \alpha_{ij}}{\partial \mathcal{R}_k} = [H_i^{-1}(\alpha_i)]_{jk} \label{eq-19}
	\end{align}
	where $H^{-1}_i$ is the inverse of the Hessian matrix of $a_i$ and where $[\cdot]_{jk}$ denotes its element $j,k$. This equation implies that if a good $k$ becomes marginally more expensive or more risky (higher $\mathcal{R}_k$), firm $i$ responds by changing its share $\alpha_{ik}$ of good $k$ by $[H_i^{-1} (\alpha_i)]{}_{kk}$. Since $a_i$ is strictly concave by Assumption 1, the diagonal elements of $H_i^{-1}$ are negative, and so a higher $\mathcal{R}_k$ always leads to a decline in $\alpha_{ik}$. The size of that decline depends on the curvature of $a_i$.
	
	\paragraph{Substitutes and Complements} 
	Whether the increase in $\mathcal{R}_k$ leads to a decline or an increase in the share of other inputs $j\neq k$ depends on whether the shares of $j$ and $k$ are complements or substitutes in the production of good $i$. If $[H_i^{-1}]_{jk} > 0$ we say that they are \textbf{substitutes}, and in that case a higher risk-adjusted price $\mathcal{R}_k$ leads to an increase in $\alpha_{ij}$. As the firm decreases $\alpha_{ik}$, the incentives embedded in $a_i$ to increase
	$\alpha_{ij}$ get stronger, and the firm substitutes $\alpha_{ij}$ for $\alpha_{ik}$. In contrast, if $[H_i^{-1}]_{jk} < 0$ we say that the
	shares of $j$ and $k$ are \textbf{complements}, and an increase in $\mathcal{R}_k$ leads to a decline in $\alpha_{ij}$. One sufficient
	condition for a Hessian matrix $H_i$ to feature complementarities for all sectors is $[H_i]_{jk} \geqslant 0 $ for all $j\neq k$.
	
	\subsection*{Example: Substitutability and complementarity in partial equilibrium}
	To show how the substitution patterns embedded in ai affect technique choices, we can revisit the car manufacturer example from the introduction. Suppose that this manufacturer primarily uses steel (input 1) to produce cars, and that it relies on equipment (input 2) such as milling machines and lathes to transform raw steel into usable components. As before, the manufacturer also has the option to purchase carbon fiber (input 3) to replace steel components if needed. It would be natural to endow this manufacturer (sector $i = 4$) with a TFP shifter function of the form
	\begin{align}
		a_4(\alpha_4) = -\sum\limits_{j=1}^4 \kappa_j (\alpha_{4j} - \alpha_{4j}^{\circ})^2 - \psi_1(\alpha_{41}-\alpha_{42})^2 - \psi_2[(\alpha_{41} + \alpha_{43}) - (\alpha_{41}^{\circ} + \alpha_{43}^{\circ})]^2,
	\end{align}
	where $\kappa_j>0$, $\psi_1>0$ and $\psi_2>0$. From the second term, we see that any increase in the share $\alpha_{41}$ of steel would incentivize the firm to increase the share $\alpha_{42}$ of steel machinery as well. Inputs 1 and 2 are therefore complements in the production of cars. In contrast, the third term implies that any increase in the share $\alpha_{41}$ of steel would make it optimal to reduce the share $\alpha_{43}$ of carbon fiber, and so the shares of inputs 1 and 3 are substitutes. These patterns can be confirmed by computing the inverse Hessian of $a_4$ directly and inspecting the off-diagonal terms. The parameters $\psi_1>0$ and $\psi_2 > 0$ determine the strength of these substitution-complementarity patterns.
	
	Figure 1 shows what happens to the production technique chosen by this car manufacturer if the risk-adjusted price of steel increases. In panel (a) the increase in $\mathcal{R}_1$ comes from a higher expectedprice $\mathbb{R}[p_1]$, while in panel (b) the price of steel becomes more volatile (higher $\mathbb{V}[p_1]$). Naturally, when the risk-adjusted price of steel rises, the manufacturer relies less on steel in production, and $\alpha_{41}$ falls. Since steel machinery is only useful when steel is used in production, the share $\alpha_{42}$ falls as well. If the increase in $\mathcal{R}_1$ is large enough, the manufacturer severs the link with its steel and steelmachinery suppliers completely so that both $\alpha_{41} = \alpha_{42} = 0$. At the same time, as steel becomes more expensive in ris-adjusted terms, the firm finds a carbon fiber supplier and progressively increases the share $\alpha_{i3}$.
	
	\begin{figure}[ht]
		\caption{Impact of rising the risk-adjusted price of steel}
		\includegraphics[width=\textwidth, trim={2cm 0cm 2cm 0cm},clip]{\basepath/fig1/1.eps}
		\label{fig:1}
	\end{figure}
	
	\section{Equilibrium existence, uniqueness and efficiency}
	
	\subsection{The efficient allocation}
	\begin{lemma}
		An efficient production network $\alpha^*$ solves
		\begin{align*}
			\mathcal{W} := \max\limits_{\alpha\in\mathcal{A}} W(a,\mu,\Sigma)
		\end{align*}
		where $\mathcal{W}$ is a measure of the welfare of the household, and where
		\begin{align}
			W(a,\mu,\Sigma) := \mathbb{E}[y(\alpha)] - \frac{1}{2}(\eqnmarkbox[red]{1}{\rho} - 1) \mathbb{V}[y(\alpha)]\label{eq-21}
		\end{align}
		\annotate[yshift=-0.5em]{below,right}{1}{Risk aversion parameter}
		\noindent is a welfare under a given network $\alpha$.
	\end{lemma}
	
	\subsection*{Recasting household welfare in terms of Domar weights}
	Since Domar weights play a crucial role in determining the expected value and the variance of GDP, it will be useful to recast the problem of the social planner in the space of $\omega$. Using (\ref{eq-14}), we can write the objective function (\ref{eq-21}) as
	\begin{align*}
		\color{blue} W(a,\mu,\Sigma) := \mathbb{E}[y(\alpha)] - \frac{1}{2}(\rho - 1) \mathbb{V}[y(\alpha)] = \omega(\alpha)^T(\mu + a(\alpha)) - \frac{1}{2}(\rho-1) \omega(\alpha)^T\Sigma\omega(\alpha) 
	\end{align*}
	\textcolor{blue}{Thus we conclude that:}
	\begin{align}
		\omega^T\mu + \omega^Ta(\alpha) - \frac{1}{2}(\rho-1)\omega^T\Sigma\omega \label{eq-22}
	\end{align}
	
	The only term in this expression that does not depend exclusively on $\omega$ is $\omega^Ta(\alpha)$, which corresponds to the contribution of the TFP shifter functions $(a_1,\cdots,a_n)$ to aggregate TFP. We want to write this object in terms of $\omega$ alone. For that purpose, notice that several networks α areconsistent with a given Domar weight vector $\omega$, but that not all of them are equivalent in termsof welfare. Indeed, to achieve a given $\omega$ the planner will only select the network $\alpha$ that maximizes welfare, which amounts to maximizing $\omega^Ta(\alpha)$.
	
	Formally, consider the optimization problem
	\begin{align}
		\bar{a}(\omega) := \max\limits_{\alpha\in\mathcal{A}} w^Ta(\alpha) \label{eq-23}
	\end{align}
	
	subject to the definition of the Domar weights given by $\omega^T = \beta^T\mathcal{L}(\alpha)$. We refer to the value function $\bar{a}$ as the aggregate TFP shifter function. It provides the maximum value of TFP $\omega^Ta(\alpha)$ that can be achieved under the constraint that the Domar weights must be equal to some given vector $\omega$. We denote by $\alpha(\omega)$ the solution to (\ref{eq-23}). Since both $\bar{a}(\omega)$ and $\alpha(\omega)$ depend exclusively on the TFP shifter functions $(a_1,\cdots,a_n)$ and on the preference vector $\beta$, these two functions will be invariant, for a given $\omega$, to the changes in beliefs $(\mu, \Sigma)$ that we consider in the next sections.
	
	\subsection*{Example.} 
	We can solve explicitly for $\bar{a}(\omega)$ and $\alpha(\omega)$ under the quadratic TFP shifter function specified in (\ref{eq-2}). At an interior solution $\alpha\in \text{int} \mathcal{A}$, the optimal production network $\alpha(\omega)$ that solves (\ref{eq-23}) for a given vector of Domar weights $\omega$ is
	\begin{align}
		\alpha_i(\omega) - \alpha_i^{\circ} = H_i^{-1} \left(\sum\limits_{j=1}^n \omega_j H_j^{-1}\right)^{-1} \left(\omega - \beta - \sum\limits_{j=1}^n \omega_j \alpha_j^{\circ}\right), \label{eq-24}
	\end{align}
	for all $i$, and the associated value function $\bar{a}$ is
	\begin{align}
		\bar{a}(\omega) = \frac{1}{2}\sum\limits_{i=1}^n\omega_i(\alpha_i(\omega) - \alpha_i^{\circ})^T H_i (\alpha_i(\omega) - \alpha_i^{\circ}). \label{eq-25}
	\end{align}
	
	
	\begin{cor}
		The efficient Domar Weight vector $\omega^*$ solves
		\begin{align}
			\mathcal{W} = \max\limits_{w\in\mathcal{O}} \underbrace{\omega^T\mu + \bar{a}(\omega)}\limits_{\mathbb{E}[y]} - \frac{1}{2}(\rho -1)\underbrace{\omega^T\Sigma\omega}\limits_{\mathbb{V}[y]} \label{eq-26}
		\end{align}
		where $\mathcal{O} = \{\omega\in\mathbb{R}_+^n:\ \omega\geqslant\beta\ \text{and}\ 1\geqslant \omega^T(\mathbbm{1}-\bar{\alpha}) \}$ and $\bar{a}(\omega)$ is given by (\ref{eq-23})
	\end{cor}
	
	\begin{lemma}
		The objective function of the planner's problem (\ref{eq-26}) is strictly concave. Furthermore, there is a unique vector of Domar weights $\omega^*$ that solves that problme, and there is a unique production network $\alpha(\omega^*)$ associated with that solution.
	\end{lemma}
	
	\subsection{Fundamental properties of the equilibrium}
	\begin{prop}
		There exists a unique equilibrium, and it is efficient.
	\end{prop}
	
	
	\newpage
	\section{Beliefs and the production network}
	In this section, we characterize how beliefs $(\mu,\Sigma)$ affect the equilibrium production network. We begin with a general result that describes how a change in a sector’s risk or expected TFP impacts its own Domar weight. We then provide an expression that characterizes how the full vector ofDomar weights responds to a marginal change in $(\mu,\Sigma)$. Finally, we investigate how beliefs affect the structure of the underlying production network $\alpha$. As we only consider the equilibrium networkfrom now on, we lighten the notation by dropping the superscript $*$ when referring to equilibrium variables.
	
	\subsection{Domar weights}
	In contrast, when the network isendogenous, they are equilibrium objects that vary with $(\mu,\Sigma)$. The next proposition describes the relationship between these quantities.
	
	\begin{prop}
		The Domar weight $\omega_i$ of sector $i$ is (weakly) increasing in $\mu_i$ and (weakly) decreasing in $\Sigma_{ii}$.
	\end{prop}
	
	\subsection*{Risk-adjusted productivity shocks}
	Proposition 2 describes how the Domar weight of a sector responds to a change in its own TFP process, and it holds generally. At an interior equilibrium, we can also characterize how any change in beliefs affects the full vector $\omega$. For that purpose, we introduce a risk-adjusted version of the productivity vector $\varepsilon$ defined as
	\begin{align}
		\mathcal{E} = \underbrace{\mu}\limits_{\mathbb{E}[\varepsilon]} - \underbrace{(\rho-1)\Sigma\omega}\limits_{\text{Cov}[\varepsilon,\lambda]} \label{eq-27}
	\end{align}
	
	The vector $\mathcal{E}$ captures how higher exposure to the productivity process $\varepsilon_i$ affects the representative household’s utility. It depends on how productive each sector $i$ is in expectation, and on how its $\varepsilon_i$ covaries with the stochastic discount factor $\lambda$. If we denote by $\mathbbm{1}_i$ the column vector with $a$ $1$ as $i$th element and zeros elsewhere, we can write
	\begin{align}
		\frac{\partial \mathcal{E}}{\partial\mu_i} = \mathbbm{1}_i,\label{eq-28}
	\end{align}
	such that an increase in $\mu_i$ makes sector $i$ more attractive. It however leaves the risk-adjusted TFP of other sectors unchanged. Similarly, for a change in $\Sigma_{ij}$, we can compute
	\begin{align}
		\frac{\partial \mathcal{E}}{\partial\Sigma_{ij}} = - \frac{1}{2}(\rho-1) (\omega_j\mathbbm{1}_i + \omega_i\mathbbm{1}_j) \label{eq-29} 
	\end{align}
	
	Using the definition of $\mathcal{E}$, we can write the first-order condition of the planner’s problem (\ref{eq-26}) at an interior solution as
	\begin{align}
		\nabla \bar{a}(\omega) + \mathcal{E} = 0 \label{eq-30}
	\end{align}
	where $\nabla$ is the gradient of the aggregate TFP shifter function $\bar{a}$. This first-order condition shows that the planner balances the benefit of a sector in terms of risk-adjusted TFP against its impact on the aggregate TFP shifter.
	
	\begin{prop}
		Let $\gamma$ denote either the mean $\mu_i$ or an element of the covariance matrix $\Sigma_{ij}$. If $\omega \in \text{int} \mathcal{O}$, then the response of the equilibrium Domar weights to a change in $\gamma$ is given by
		\begin{align}
			\frac{d\omega}{d\gamma} = \underbrace{-\mathcal{H}^{-1}}\limits_{\text{propagation}} \times \underbrace{\frac{\partial\mathcal{E}}{\partial\gamma}}\limits_{\text{impulse}} \label{eq-31}
		\end{align}
		where the $n\times n$ negative definite matrix $\mathcal{H}$ is given by
		\begin{align}
			\mathcal{H} = \nabla^2\bar{a} + \frac{\partial\mathcal{E}}{\partial\omega} \label{eq-32}
		\end{align}
		and where the matrix $ \nabla^2\bar{a}$ is the Hessian of the aggregate TFP shifter function $\bar{a}$, and $\frac{\partial\mathcal{E}}{\partial\omega} = - \frac{d\text{Cov}[\varepsilon,\lambda]}{d\omega} = - (\rho-1)\Sigma$ is the Jacobian matrix of the risk-adjusted TFP vector $\mathcal{E}$.
	\end{prop}
	
	The response of the Domar weights to a change in beliefs, as given by (\ref{eq-31}), can be decomposed into an impulse component and a propagation component. The impulse captures the direct impact of the change on risk-adjusted TFP. It is simply given by the partial derivative of $\mathcal{E}$ with respect to the moment of interest (see (\ref{eq-28}) and (\ref{eq-29}) above). This impulse is then propagated through $\mathcal{H}^{-1}$ to capture its full equilibrium effect on the Domar weights.
	
	\paragraph{Global complements and substitutes}
	Just as $\mathcal{H}_i^{-1}$ captured local substitution patterns between inputs in the problem of firm $i$, $\mathcal{H}^{-1}$
	captures global, economy-wide substitution patterns between sectors. If $\mathcal{H}_{ij}^{-1}<0$, we say that $i$ and $j$ are \textbf{global complements}. If instead $\mathcal{H}_{ij}^{-1}>0$, we say that $i$ and $j$ are \textbf{global substitutes}.
	
	The following corollary justifies this terminology by showing that the sign of $\mathcal{H}_{ij}^{-1}$ is sufficient to characterize how Domar weights respond to a change in the productivity process.
	
	\begin{cor}
		If $w\in\text{int}\mathcal{O}$, then the following holds.
		\begin{enumerate}[leftmargin=1cm, label=\arabic*.]
			\item An increase in the expected value $\mu_i$ or a decline in the variance $\Sigma_{ii}$ leads to an increase in $\omega_j$ if $i$ and $j$ are global complements, and to a decline in $\omega_j$ if $i$ and $j$ are global substitutes.
			\item An increase in the covariance $\Sigma_{ij}$, $i \neq j$, leads to a decline in $\omega_k$ if $k$ is global complement
			with $i$ and $j$, and to an increase in $\omega_k$ if $k$ is global substitute with $i$ and $j$.
		\end{enumerate}
	\end{cor}
	
	
	\paragraph{$\Sigma$ and global substition patterns}
	The following lemma describes how an increase in covariance $\Sigma_{ij}$ between any two sectors affects the degree of global substitution between them.
	
	\begin{lemma}
		An increase in the covariance $\Sigma_{ij}$ induces stronger global substitution between $i$ and $j$, in the sense that $\frac{\partial \mathcal{H}_{ij}^{-1}}{\partial\Sigma_{ij}}>0$.
	\end{lemma}
	
	Intuitively, if the correlation between $\varepsilon_i$ and $\varepsilon_j$ becomes larger, the planner has stronger incentives to lower $\omega_j$ after an increase in $\omega_i$ in order to reduce aggregate risk. From (\ref{eq-32}), we see that the strength of that diversification mechanism depends on the household’s risk aversion through $\rho$.
	
	\paragraph{$\nabla^2\bar{a}$ and global substitution patterns}
	The next lemma establishes sufficient conditions under which local complementarities translate into global complementarities.
	
	\begin{lemma}
		Suppose that all input shares are (weak) local complements in the production of all goods, that is $[H_i^{-1}]_{kl} \leqslant 0$ for all $i$ and all $k\neq l$. If $\alpha\in\text{int}\mathcal{A}$, there exists a scalar $\bar{\Sigma}>0$ such that if $\|\Sigma\| \leqslant \bar{\Sigma}$, all sectors are global complements, that is $\mathcal{H}_{ij}^{-1} < 0$ for all $i\neq j$.
	\end{lemma}
	
	\paragraph{Impact of Lemma 6}
	\textcolor{blue}{\begin{enumerate}[leftmargin=1cm, label=\arabic*.]
			\item \textbf{Generation of Global Complementarities}: Even if the local TFP shifter functions are neutral (i.e., \(\left[ H_i^{-1} \right]_{kl} = 0\) for all \(i\) and \(k \neq l\)), the equilibrium forces of the model generate global complementarities between sectors. This means that the model itself induces sectors to be globally complementary without requiring local TFP shifter functions to exhibit local complementarities.
			\item \textbf{Equilibrium Forces} Suppose a sector \(i\) becomes more attractive, for instance due to an increase in \(\mu_i\). Any other sector \(j\) that relies on \(i\) (either directly or indirectly, if \(L_{ji} > 0\)) would benefit from that change and also become more attractive. This triggers an increase in Domar weights throughout the network and a shift away from labor, generating global complementarities between sectors.
			\item \textbf{Policy and Practical Applications} Understanding the conditions under which local complementarities translate into global complementarities can help in formulating more effective economic policies, especially regarding resource allocation and inter-sector coordination. This is crucial for improving overall economic efficiency and welfare.
			\item \textbf{Role of Covariance Matrix (\(\Sigma\))} The lemma highlights that the degree of global substitution or complementarity between sectors can be influenced by the covariance matrix \(\Sigma\). If \(\Sigma\) is sufficiently small, local complementarities can lead to global complementarities, while larger \(\Sigma\) might induce stronger global substitution forces due to diversification effects.
	\end{enumerate}}

	\paragraph{Parametrize $H_i$}
	Let
	\begin{align}
		H_i^{-1} = \left[\begin{array}{cccc}
			-1 & \frac{s}{n-1} & \cdots & \frac{s}{n-1} \\
			\frac{s}{n-1} & -1 &  & \vdots \\
			\vdots & & \ddots & \frac{s}{n-1} \\
			\frac{s}{n-1} & \cdots & \frac{s}{n-1} & -1 \\
		\end{array} \right] \label{eq-33}
	\end{align}
	where we impose $-(n-1) < s < 1$ to guarantee that $H_i^{-1}$ is negative definite. When $s < 0$ all input shares are complements in the production of good $i$, and when $s > 0$ they are substitutes. The next lemma describes sufficient conditions under which local substitution imply global substitution.
	
	\begin{lemma}
		Suppose that all the TFP shifter functions $(a_1, \cdots ,a_n)$ take the form (\ref{eq-2}), with $\alpha_i^{\circ} = \alpha_j^{\circ}$ for all $i, j$, and that $H_i^{-1}$ is of the form (\ref{eq-33}) for all $i$. If $\alpha\in\text{int}\mathcal{A}$, there exists a scalar $\bar{\Sigma} > 0$ and a threshold $0 < \bar{s} < 1$ such that if $\|\Sigma\| \leqslant \bar{\Sigma}$ and $s > \bar{s}$, then all sectors are global substitutes, that is $\mathcal{H}_{ij}^{-1} > 0$ for all $i\neq j$.
	\end{lemma}
	
	\paragraph{An approximate equation for the equilibrium Domar weights}
	This section discusses how to derive an approximate equation for the equilibrium Domar weights using a Taylor expansion of \(\nabla\bar{a}\). The key steps and impacts are outlined as follows:
	
	First, we define the ideal shares \( \alpha^\circ \), which maximize the values of the TFP shifters \( (a_1, \dots, a_n) \). Based on this, we can write:
	\begin{align}
		\nabla \bar{a} (\omega) \approx \nabla \bar{a} (\omega^\circ) + \nabla^2 \bar{a} (\omega^\circ) (\omega - \omega^\circ) \label{eq-34}
	\end{align}
	
	This approximation is accurate if the cost of deviating from the ideal shares embedded in the local TFP shifters is large.
	
	Using this approximation, the first-order condition (\ref{eq-30}) becomes linear in \( \omega \), allowing us to solve for the equilibrium Domar weights.
	
	\begin{lemma}
		If \( \omega \in\text{int}\mathcal{O} \), the equilibrium Domar weights are approximately given by:
		\begin{align}
			\omega = \omega^\circ - [\mathcal{H}^\circ]^{-1} \mathcal{E}^\circ + O\left(\|\omega - \omega^\circ\|^2\right) \label{eq-35}
		\end{align}
		where the superscript \( \circ \) indicates that \( \mathcal{H} \) and \( \mathcal{E} \) are evaluated at \( \omega^\circ \).
	\end{lemma}
	
	\paragraph{Impacts of Lemma 8}
	\textcolor{blue}{\begin{enumerate}[leftmargin=1cm, label=\arabic*.]
		\item \textbf{Global Substitution Patterns} This approximation shows that the equilibrium Domar weights can be explained in terms of the global substitution patterns embedded in \( [ \mathcal{H}^\circ]^{-1} \) and the expected attractiveness of all sectors, captured by the risk-adjusted productivity \(  \mathcal{E}^\circ \).
		\item \textbf{Inter-Sector Interactions} If a sector \( i \) is endowed with a productivity process that is high in expectation or has a low covariance with the stochastic discount factor, \(  \mathcal{E}^\circ_i \) will be large. Since the diagonal elements of \( [ \mathcal{H}^\circ]^{-1} \) are negative, \( \omega_i \) tends to be larger than \( \omega^\circ_i \).
		\item \textbf{Relative Weight Changes} A large \( \mathcal{E}^\circ_i \) also contributes to increasing the Domar weights of all sectors that are global complements with \( i \) and to decreasing the Domar weights of sectors that are global substitutes with \( i \).
	\end{enumerate}}

	
	\subsection{The production network}
	\begin{prop}
		If $\alpha\in\text{int}\mathcal{A}$, there exists a scalar $\bar{\Sigma}>0$ such that if $\|\Sigma\|\leqslant\bar{\Sigma}$ the following holds.
		\begin{enumerate}[leftmargin=1cm, label=\arabic*.]
			\item (Complementarity) Suppose that input shares are local complements in the production of good $i$, that is $[H_i^{-1}]_{kl} < 0$ for all $k\neq l$. Then a beneficial change to $k$ ($\partial \mathcal{E}_k / \partial \gamma > 0$) increases $\alpha_{ij}$ for all $j$.
			\item (Substitution) Suppose that the conditions of Lemma 7 about the TFP shifters $(a_1,\cdots,a_n)$ hold. Then there exists a threshold $0<\bar{s}<1$ such that if $s > \bar{s}$, a beneficial change to $k$ ($\partial \mathcal{E}_k / \partial \gamma > 0$) decreases $\alpha_{ij}$ for all $i$ and all $j \neq k$, and increases $\alpha_{ik}$ for all $i$.
		\end{enumerate}
	\end{prop}
	
	\textcolor{blue}{Proposition 4 illustrates the impact of complementarity and substitution of input shares on the adjustment of production networks. When input shares are locally complementary in the production of a product, a beneficial change to one input increases its share in the production of all products. Conversely, in the presence of strong substitution effects, a beneficial change to one input decreases the shares of other inputs in production while increasing its own share.}
	
	\subsection*{An approximate equation for the equilibrium production network}
	As for the Domar weights, one must in general use numerical methods to find the equilibrium network $\alpha$. We can, however, derive an approximation for the equilibrium production network when the cost of deviating from the ideal shares $\alpha^{\circ}$ is large. Specifically, let $a_i (\alpha_i) = \bar{\kappa} \times \hat{a}_i (\alpha_i)$, where $\hat{\alpha}$ does not depend on $\kappa$, and suppose that $\alpha_i^{\circ}\in\text{int}\mathcal{A}_i$. The parameter $\hat{\kappa}>0$ captures how costly it is for the firms to deviate from $\alpha^{\circ}$ in terms of TFP loss. When $\hat{\kappa}$ is large, we can use perturbation theory to derive an approximate equation for $\alpha$.
	
	\begin{lemma}
		If $\alpha\in\text{int}\mathcal{A}$, the equilibrium input shares in sector $i$ are approximately given by
		\begin{align}
			\alpha_i = \alpha_i^{\circ} + \bar{\kappa}^{-1} \left(\hat{H}_i^{\circ}\right)^{-1} \mathcal{R}^{\circ} + O(\kappa^{-2}) \label{eq-36}
		\end{align}
		where $\hat{H}_i^{\circ}$ is the Hessian of $\hat{a}_i$ at $\alpha_i^{\circ}$, and where the vector of risk-adjusted prices at $\alpha^{\circ}$ is given by
		\begin{align*}
			\mathcal{R}^{\circ} = -\mathcal{L}\mu + (\rho - 1)\mathcal{L}^{\circ}\Sigma\omega^{\circ}
		\end{align*}
	\end{lemma}

	Lemma 9 primarily addresses the approximate solution for the production network when the cost function is nonlinear. Specifically, when it is costly for firms to deviate from the ideal shares \(\alpha^\circ\), the equilibrium production network can be approximated using perturbation theory. Equation (\ref{eq-36}) provides an approximation indicating that the equilibrium input shares \(\alpha_i\) depend on the risk-adjusted prices \(\mathcal{R}^\circ\). This result demonstrates that when the cost of deviating from the ideal shares is high, the equilibrium production network can be approximated by evaluating the equilibrium prices as if firms chose the ideal shares.

	\subsection*{Example: cascading link destruction}
	The example of ``cascading link destruction'' discusses how an increase in uncertainty in a single sector can trigger a chain reaction throughout the production network. Specifically, when the volatility of a sector increases, multiple producers sequentially switch to more stable suppliers, causing a series of adjustments in the production network.
	
	The specific example is as follows:
	\begin{enumerate}[leftmargin=1cm, label=\arabic*.]
		\item In a low-uncertainty state (left figure): Firms in sectors 1 to 3 directly or indirectly rely on sector 4 as a supplier.
		\item In a high-uncertainty state (right figure): As the uncertainty in sector 4 increases, firms in sector 3, seeking a more stable supply, switch to using inputs from sector 7. This change implies that firms in sector 2, to avoid risk, switch to using inputs from sector 6, and so on, creating a cascade of adjustments.
	\end{enumerate}

	Through this example, the paper demonstrates how the production network adjusts in response to changes in uncertainty. These adjustments not only affect the directly related firms but also propagate through the supply chain, impacting firms far removed from the initial shock.
	
	\begin{figure}[ht]
		\caption{Cascading impact of a change in $\Sigma_{44}$}
		\includegraphics[width=\textwidth]{\basepath/fig2/1.png}
		\label{fig:2}
	\end{figure}
	
	We can interpret this cascading network adjustment through the lens of Lemma 9. Differentiating the expression with respect to $\Sigma_{44}$ yields
	\begin{align}
		\frac{d\alpha_{ij}}{d\Sigma_{44}} = \bar{\kappa}^{-1} (\rho - 1) \omega_4^{\circ} \left( \underbrace{\left[ \left(\hat{H}_i^{\circ}\right)^{-1} \right]_{jj} \mathcal{L}^{\circ}_{j4}}\limits_{\text{direct effect of }\Sigma_{44}\text{ on }j} + \underbrace{\sum\limits_{l\neq j} \left[\left(\hat{H}_i^{\circ}\right)^{-1}\right]_{jl}\mathcal{L}^{\circ}_{jl}}_{\text{indirect effect of }\Sigma_{44}\text{ through other suppliers }l\neq j}
		 \right) + O(\bar{\kappa}^{-2}) \label{eq-37}
	\end{align}
	
	Equation (\ref{eq-37}) states that if a firm $j$ relies on sector 4 as an input (either immediate or distant, such that $\mathcal{L}_{j4}^{\circ} > 0$), an increase in $\Sigma_{44}$ makes $j$ less attractive. This direct effect pushes $\alpha_{ij}$ down (recall
	that $[H_i^{\circ}]_{jj} < 0$ by the concavity of $a_i$). There is also an indirect effect that operates through the second term in (\ref{eq-37}). If another sector $l\neq j$ also relies on 4 ($\mathcal{L}_{l4}^{\circ} > 0$), then an increase in $\Sigma_{44}$
	makes $l$ less attractive as well. This indirect channel can lead to either a decrease or an increase in $\alpha_{ij}$, depending on whether $j$ and $l$ are complements or substitutes in the production of $i$; that is, whether $[(H_i^{\circ})^{-1}]_{jl}$ is negative or positive.
	
	\section{Implications for GDP and welfare}
	\begin{prop}
		Let $\gamma$ denote either the mean $\mu_i$ or an element of the covariance matrix $\Sigma_{ij}$. Under an endogenous network, welfare responds to a marginal change in $\gamma$ as if the network were fixed at its equilibrium value $\alpha^*$, that is
		\begin{align*}
			\frac{d\mathcal{W}(\mu,\Sigma)}{d\gamma} = \frac{\partial W(\alpha^*,\mu,\Sigma)}{\partial\gamma}
		\end{align*}
	\end{prop}
	
	\textcolor{blue}{Let \(\alpha^*\) be the equilibrium network, i.e., \(\alpha^* = \alpha(\mu, \Sigma)\). When we make a small change to \(\gamma\), the equilibrium network will adjust to accommodate the new \(\gamma\). However, Proposition 5 states that the effect of this adjustment on the marginal change can be neglected.}

	While this proposition shows that the flexibility of the network plays no role for the response of welfare to a marginal change in beliefs, this is generally not true for non-infinitesimal changes. In that case, shifts in $(\mu, \Sigma)$ that are beneficial to welfare are amplified, compared to the fixed-network benchmark, while changes that are harmful are dampened (see Proposition 2). Indeed, if we denote by $\alpha^*(\mu,\Sigma)$ the equilibrium production network under $(\mu, \Sigma)$ and by $W(\alpha,\mu,\Sigma)$ welfare under a network $\alpha$, we can write that the difference in welfare after a change in beliefs from $(\mu,\Sigma)$ to $(\mu^{\prime},\Sigma^{\prime})$ satisfies the inequality
	\begin{align}
		\underbrace{\mathcal{W}(\mu^{\prime},\Sigma^{\prime}) - \mathcal{W}(\mu,\Sigma)}\limits_{\text{Change in welfare under a flexible network}} \geqslant \underbrace{W(\alpha^*(\mu,\Sigma),\mu^{\prime},\Sigma^{\prime}) - W(\alpha^*(\mu,\Sigma),\mu,\Sigma)}\limits_{\text{Change in welfare under a fixed network}}. \label{eq-38}
	\end{align}
	
	\begin{cor}
		The impact of an increase in $\mu_i$ on welfare is given by
		\begin{align}
			\frac{d\mathcal{W}}{d\mu_i} = \omega_i \label{eq-39}
		\end{align}
		and the impact of an increase in $\Sigma_{ij}$ on welfare is given by
		\begin{align}
			\frac{d\mathcal{W}}{d\Sigma_{ij}} = -\frac{1}{2}(\rho-1)\omega_i\omega_j
		\end{align}
	\end{cor}
	
	\subsection{Beliefs and GDP}
	\begin{prop}
		The presence of uncertainty lowers expected log GDP, in the sense that $\mathbb{E}[y]$ is largest when $\Sigma = 0$.
	\end{prop}
	
	This proposition follows directly from Lemma 3. Without uncertainty $(\Sigma = 0)$, the variance $\mathbb{V} [y]$ of log GDP is zero for all networks $\alpha\in\mathcal{A}$. The social planner then maximizes $\mathbb{E}[y]$ only. When, instead, the productivity vector $\varepsilon$ is uncertain $(\Sigma \neq 0)$, the planner also seeks to lower $\mathbb{V} [y]$ which necessarily lowers expected log GDP in equilibrium.
	
	
	\begin{cor}
		Let $\gamma$ denote either the mean $\mu_i$ or an element of the covariance matrix $\Sigma_{ij}$. The equilibrium response to a change in beliefs $\gamma$ must satisfy
		\begin{align}
			\underbrace{\frac{d\mathbb{E}[y]}{d\gamma} - \frac{\partial\mathbb{E}[y]}{\partial \gamma}}\limits_{\text{Excess response of }\mathbb{E}[y]} = \frac{1}{2}(\rho - 1)\underbrace{\left(\frac{d\mathbb{V}[y]}{d\gamma} - \frac{\partial \mathbb{V}[y]}{\partial\gamma}\right)}\limits_{\text{Excess response of }\mathbb{V}[y]} \label{eq-41}
		\end{align}
	\end{cor}
	
	Corollary 5 is a direct consequence of Proposition 5. Since the response of welfare to a marginal change in beliefs must be the same under a flexible and a fixed network, a larger increase in $\mathbb{E}[y]$ under a flexible network must come at the cost of alarger increase in the variance $\mathbb{V}[y]$. 
	
	\begin{prop}
		If $\omega\in\text{int}\mathcal{O}$, the following holds.
		\begin{enumerate}[leftmargin=1cm, label=\arabic*.]
			\item The impact of an increase in $\mu_i$ on log GDP is given by
			\begin{align*}
				\frac{d\mathbb{E}[y]}{d\mu_i} = \underbrace{\omega_i}\limits_{\text{Fixed network}} - (\rho-1)\omega^T\Sigma\mathcal{H}^{-1}\frac{\partial\mathcal{E}}{\partial\mu_i}, \quad \text{and}\quad \frac{d\mathbb{V}[y]}{d\mu_i} = \underbrace{0}\limits_{\text{Fixed network}} - 2\omega^T\Sigma\mathcal{H}^{-1}\frac{\partial\mathcal{E}}{\partial\mu_i}. 
			\end{align*}
			\item The impact of an increase in $\Sigma_{ij}$ on log GDP is given by
			\begin{align*}
				\frac{d\mathbb{E}[y]}{d\Sigma_{ij}} = \underbrace{0}\limits_{\text{Fixed network}} - (\rho-1)\omega^T\Sigma\mathcal{H}^{-1}\frac{\partial\mathcal{E}}{\partial\Sigma_{ij}}, \quad \text{and}\quad \frac{d\mathbb{V}[y]}{d\Sigma_{ij}} = \underbrace{\omega_i\omega_j}\limits_{\text{Fixed network}} - 2\omega^T\Sigma\mathcal{H}^{-1}\frac{\partial\mathcal{E}}{\partial\Sigma_{ij}}. 
			\end{align*}
		\end{enumerate}
	\end{prop}
	
	\begin{cor}
		Without uncertainty $(\Sigma = 0)$ the moments of GDP respond to changes in beliefs as if the network were fixed, such that
		\begin{align*}
			\frac{d\mathbb{E}[y]}{d\mu_i} = \frac{\partial\mathbb{E}[y]}{\partial\mu_i} = \omega_i, \quad \text{and} \quad \frac{d\mathbb{V}[y]}{d\Sigma_{ij}} = \frac{\partial\mathbb{V}[y]}{\partial\Sigma_{ij}} = \omega_i\omega_j
 		\end{align*}
	\end{cor}
	
	\begin{cor}
		Suppose that $\omega\in\text{int}\mathcal{O}$. There exists a threshold $\bar{\Sigma} < 0$ such that if $\Sigma_{kl} > \bar{\Sigma}$ for all $k, l$, then the following holds.
		\begin{enumerate}[leftmargin=1cm, label=\arabic*.]
			\item If all sectors are global complements with sector $i$, that is $\mathcal{H}_{ik}^{-1} < 0$ for $k\neq i$, then
			\begin{align*}
				\frac{d\mathbb{E}[y]}{d\mu_i} = \frac{\partial\mathbb{E}[y]}{\partial\mu_i} > \omega_i, \quad \text{and} \quad \frac{d\mathbb{V}[y]}{d\Sigma_{ij}} = \frac{\partial\mathbb{V}[y]}{\partial\mu_i} > 0
			\end{align*}
			\item If all sectors are global complements with sectors $i$ and $j$, that is $\mathcal{H}_{ik}^{-1} < 0$ and $\mathcal{H}_{jk}^{-1}$ for $k \neq i, j$, then
			\begin{align*}
				\frac{d\mathbb{E}[y]}{d\mu_i} = \frac{\partial\mathbb{E}[y]}{\partial\Sigma_{ij}} < 0, \quad \text{and} \quad \frac{d\mathbb{V}[y]}{d\Sigma_{ij}} = \frac{\partial\mathbb{V}[y]}{\partial\mu_i} < \omega_i\omega_j
			\end{align*}
		\end{enumerate}
	\end{cor}
	
	
	
	
	
	
\end{document}